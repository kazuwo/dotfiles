\documentclass[11pt,a4paper]{jsarticle}
%
\usepackage{amsmath,amssymb}
\usepackage{bm}
\usepackage{graphicx}
\usepackage{ascmac}
%
\setlength{\textwidth}{\fullwidth}
\setlength{\textheight}{39\baselineskip}
\addtolength{\textheight}{\topskip}
\setlength{\voffset}{-0.5in}
\setlength{\headsep}{0.3in}
%
\pagestyle{myheadings}
\markright{\footnotesize \sf vimコマンドメモ \ \texttt{ 2016/11/14 }}
%
\begin{document}
%
\section{ファイル操作}
:e.	エクスプローラーを開く
    このとき、ファイル名の上でpを押したら内容が確認できる。そのままファイルを編集したい場合は<Ctrl>+w

ワイルドカードで開く
:e **/(ファイル名) <Tab>	Tabを押すたびに:e 以降にパスの候補を表示
例1::e **/hoge.text	path以下のすべての階層からhoge.txtを探して候補に表示
例2::e application/**/*.text	application以下のすべての階層から.txtを探して候補に表示

:sp <ファイルパス>	現在のウィンドウを横に分割してファイルを開く
:vp <ファイルパス>	現在のウィンドウを縦に分割してファイルを開く
Ctr+w+w	次のウィンドウへフォーカスを移動
Ctr+w+↑(k)	上のウィンドウへフォーカスを移動
Ctr+w+↓(j)	下のウィンドウへフォーカスを移動
Ctr+w+-	現在のウィンドウを広げる
Ctr+w+=	全てのウィンドウの大きさを揃える

\section{移動}
f[文字]	次の[文字]まで飛ぶ
F[文字]	前の[文字]まで飛ぶ
t[文字]	次の[文字]の手前まで飛ぶ
T[文字]	前の[文字]の次まで飛ぶ
;	f, F, t, Tで検索した文字を同方向に繰り返し検索する
,	f, F, t, Tで検索した文字を逆方向に繰り返し検索する
g;	直前に変更した位置にカーソルを移動(繰り返し入力することで, 変更リストを遡る)
H, M, L	画面に表示されている行について, 最初の行(H), 真ん中の行(M), 最後の行(L)に移動
Ctrl-u	半画面分戻る
Ctrl-d	半画面分進む
Ctrl-b	1画面分戻る
Ctrl-f	1画面分進む
括弧の上で\%を実行すると、括弧の開始位置と終了位置を行き来できます。

e ge w b	単語系(単語頭末)ごとに移動
{ } ( )	パラグラフ、文章ごとに移動
/(単語)	単語を検索。検索後、 n(次へ) 、 N(前へ)
f(文字)	行中の文字まで移動


\section{モード変更}
I	行頭からインサートモードに入る
A	行末からインサートモードに入る
Ctrl-o(インサートモード中に)	インサートノーマルモードに入る(インサートモード中に1コマンドのみノーマルモードのコマンドを使用できる)
gv	前回選択した範囲が再選択された状態でビジュアルモードに入る


\section{削除、コピー}
dt(	カーソル位置から, 次に最初に出現する(までを削除する
cc	カーソルがある行を消して, インサートモードに入る
cw	カーソル位置から1単語分削除して, インサートモードに入る
カーソル位置から, 次に最初に出現する(までを削除して, インサートモードに入る


\section{検索}
/	続けて文字を打ち, Enterで文字列を確定し, 後方検索をかける
?	続けて文字を打ち, Enterで文字列を確定し, 前方検索をかける
*	カーソル下の単語を, 後方検索する
#	カーソル下の単語を, 前方検索する
g*	カーソル下の単語を含む文字列を, 後方検索する
g#	カーソル下の単語を含む文字列を, 前方検索する
n	次の単語に移る
N	前の単語に移る
:noh  検索結果のハイライトを消したい

複数ファイルで特定文字列の検索

プログラムを書いていると、特定のクラスがどこで定義されているのか、この定数がどこで定義されているのかなど調べなければいけない場合があるかと思います。そんな場合でもVimはデフォルトで複数ファイルの特定文字列の検索機能を提供しています。
:vim(grep) 文字列 検索対象階層・ファイル | cw	特定の文字列を検索対象階層・ファイル
(例):vim hoge **/* | cw	vimを開いた階層以下のファイルをhogeという文字列で検索
(例):vim abc test/*.py | cw	「test」階層以下の.pyの拡張子のファイルをabcという文字列で検索
「| cw」はなくても検索出来るのですが、この「| cw」をつけることによって、検索結果をVimのQuickfixリスト(ファイル一覧)を開いて検索結果を一覧で表示してくれます。



\section{Undo, Redo, 繰り返し}
u	Undo
Ctrl-r	Redo
.	直前の操作の繰り返し


\section{インデント}
=	カーソルがある行をインデント. ビジュアルモードで範囲を選択していれば, その範囲をインデント
>>	インデントを追加
<<	インデントを削除.


\section{}
\section{}
\section{}
\section{}
\section{}


\if0
http://qiita.com/ykyk1218/items/ab1c89c4eb6a2f90333a
http://qiita.com/kmszk/items/9be187a47d5806b113da
http://qiita.com/kmszk/items/e532a7578e06a77296fa
\fi





%
%
\end{document}

